\documentclass[swedish]{kththesis}

% remove this if you are using XeLaTeX or LuaLaTeX
\usepackage[utf8]{inputenc}

% Use natbib abbreviated bibliography style
\usepackage[square,numbers]{natbib}
\bibliographystyle{unsrtnat}

\usepackage{lipsum} % This is just to get some nonsense text in this template, can be safely removed


\title{Detta är den svenska titeln}
\alttitle{This is the English translation of the title}
\author{Osquar Student}
\email{osquar@kth.se}
\supervisor{Lotta Larsson}
\examiner{Lennart Bladgren}
\principal{Företaget AB}
\programme{Civilingenjör Datateknik}
\school{Skolan för Datavetenskap och Kommunikation}
\date{\today}


\begin{document}

% Försättsblad
\flyleaf

\begin{abstract}
  Svensk sammanfattning.
  \lipsum[1-2]
\end{abstract}

\clearpage

\begin{otherlanguage}{english}
  \begin{abstract}
  English abstract.
  \lipsum[3-4]
  \end{abstract}
\end{otherlanguage}

\cleardoublepage

\tableofcontents


% This is where the actual contents of the thesis starts
\mainmatter


\chapter{Introduktion}

Vi använder paketet \emph{natbib} för litteraturreferenser.  Därför
anropar vi kommandot \texttt{citep} för att få referenser inom
parentes, så här \citep{heisenberg2015}.  Det är också möjligt att använda
författarens namn som en del av en mening genom att använda
\texttt{citet}, om vi t.ex.\ talar om en studie av \citet{einstein2016}.

\lipsum

\section{Frågeställning}

\lipsum[6]

\chapter{Method}

\lipsum

\bibliography{references}

\appendix

\chapter{Onödigt Material som Bilaga}

\end{document}
